% THIS IS SIGPROC-SP.TEX - VERSION 3.1
% WORKS WITH V3.2SP OF ACM_PROC_ARTICLE-SP.CLS
% APRIL 2009
%
% It is an example file showing how to use the 'acm_proc_article-sp.cls' V3.2SP
% LaTeX2e document class file for Conference Proceedings submissions.
% ----------------------------------------------------------------------------------------------------------------
% This .tex file (and associated .cls V3.2SP) *DOES NOT* produce:
%       1) The Permission Statement
%       2) The Conference (location) Info information
%       3) The Copyright Line with ACM data
%       4) Page numbering
% ---------------------------------------------------------------------------------------------------------------
% It is an example which *does* use the .bib file (from which the .bbl file
% is produced).
% REMEMBER HOWEVER: After having produced the .bbl file,
% and prior to final submission,
% you need to 'insert'  your .bbl file into your source .tex file so as to provide
% ONE 'self-contained' source file.
%
% Questions regarding SIGS should be sent to
% Adrienne Griscti ---> griscti@acm.org
%
% Questions/suggestions regarding the guidelines, .tex and .cls files, etc. to
% Gerald Murray ---> murray@hq.acm.org
%
% For tracking purposes - this is V3.1SP - APRIL 2009

\documentclass{acm_proc_article-sp}
\usepackage{url}
\begin{document}

\title{Twitter-based Sentiment Analysis}

%
% You need the command \numberofauthors to handle the 'placement
% and alignment' of the authors beneath the title.
%
% For aesthetic reasons, we recommend 'three authors at a time'
% i.e. three 'name/affiliation blocks' be placed beneath the title.
%
% NOTE: You are NOT restricted in how many 'rows' of
% "name/affiliations" may appear. We just ask that you restrict
% the number of 'columns' to three.
%
% Because of the available 'opening page real-estate'
% we ask you to refrain from putting more than six authors
% (two rows with three columns) beneath the article title.
% More than six makes the first-page appear very cluttered indeed.
%
% Use the \alignauthor commands to handle the names
% and affiliations for an 'aesthetic maximum' of six authors.
% Add names, affiliations, addresses for
% the seventh etc. author(s) as the argument for the
% \additionalauthors command.
% These 'additional authors' will be output/set for you
% without further effort on your part as the last section in
% the body of your article BEFORE References or any Appendices.

\numberofauthors{5} %  in this sample file, there are a *total*
% of EIGHT authors. SIX appear on the 'first-page' (for formatting
% reasons) and the remaining two appear in the \additionalauthors section.
%
\author{
% You can go ahead and credit any number of authors here,
% e.g. one 'row of three' or two rows (consisting of one row of three
% and a second row of one, two or three).
%
% The command \alignauthor (no curly braces needed) should
% precede each author name, affiliation/snail-mail address and
% e-mail address. Additionally, tag each line of
% affiliation/address with \affaddr, and tag the
% e-mail address with \email.
%
% 1st. author
\alignauthor Jakob Gruber\\
      0203440\\
       \email{}
% 2nd. author
\alignauthor Matthias Krug\\
      0828965\\
       \email{}
% 3rd. author
\alignauthor Stefanie Plieschnegger\\
      0926102\\
\and  % use '\and' if you need 'another row' of author names
% 4th. author
\alignauthor Christian Proske\\
	 1328245 \\
       \email{}
% 5th. author
\alignauthor Mino Sharkhawy  \\
      1025887 \\
       \email{}
}
% There's nothing stopping you putting the seventh, eighth, etc.
% author on the opening page (as the 'third row') but we ask,
% for aesthetic reasons that you place these 'additional authors'
% in the \additional authors block, viz.

% Just remember to make sure that the TOTAL number of authors
% is the number that will appear on the first page PLUS the
% number that will appear in the \additionalauthors section.

\maketitle
\begin{abstract}
Sentiment analysis has become very popular in recent years and especially
Twitter provides a lot of data to a huge amount of topics which can be
processed and classified to provide an overall opinion. However, classification
of Twitter-based data is somehow different to traditional text mining and
introduce some additional challenges. In this paper typical problems are
discussed that go along with classification of tweets. We will also shortly
discuss two popular machine learning algorithms (Naive Bayes and SVM) for
sentiment analysis and explain how a classifiers are evaluated.

\end{abstract}


% A category with the (minimum) three required fields
\category{H.3}{Information Systems}{Information Search and Retrieval}

\terms{Theory}

\keywords{Sentiment Analysis, Opinion Mining, Twitter, Classifier}

\section{Introduction} The general opinion about a specific product or service
has certainly a great influence on its reputation. People often want to know
what others think about a special product they are willing to buy, about a new
movie, or about a hotel they are going to book. But also companies may
interested in its customers' opinions, politicians may wish to receive
feedback, or social organizations may have interest in an ongoing debate.
\cite{pak2010twitter}.  The world wide web provides many ways for people to
distribute their experiences and sentiments. Machine learning algorithms make
it easier to process and evaluate those sentiments and are therefore able to
provide an overall opinion to a certain topic. This kind of analyzing is called
sentiment analysis or opinion mining \cite{liu2010sentimentanalysis,
pang2008opinion}. Clearly, there are some challenges when assessing the opinion
of people, especially when classifying microblogging services like Twitter
\footnote{\url{http://twitter.com}}. The underlying paper gives an overview
about different sentiment analysis approaches and outlines special problems
related to the classification of microblogging services. These challenges are
discussed in section~\ref{analyzingdata}. Section~\ref{twitterapi} gives some
basic information about the Twitter API. In section~\ref{preprocessing} and
\ref{classification} it is describes approaches of preprocessing and
classifying tweets. In section~\ref{related} some related work is introduced
and section~\ref{conclusion} concludes the paper.


\section{Analyzing Data} \label{analyzingdata} The analyzing process of text
and the classification whether its content is rather positive, negative or may
be considered as neutral, is the core functionality in sentiment analysis. We
are going to discuss general considerations of classifying data first of all,
before outlining additional challenges related to Twitter-based data.

\subsection{Text Mining vs. Sentiment Analysis} One would assume, that text
mining and sentiment analysis are very similar. Text mining e.g. may deal with
classifying documents by topic, which represents one of the easier tasks.
Topic-based text classification tries to match a text into a category like
sport, politics etc. and therefore topic-related words are identified to
classify a text. However, sentiment analysis requires to focus on typical
"sentiment words" for example hate, love, like, regret.  When it comes to
identify the overall sentiment of a text it turns out that it may contains
several aspects (e.g. negative and positiver remarks) or may not even contain
any "signal words" (e.g. terrible, awful, bad) but still has a negative
meaning.  To underline this problem, here are some examples:

\textit{"My new smartphone is really cool, the display is just gorgeous. The
battery life is really bad, however."}

\textsl{"Oh, of course - I have a lot of time. Just keep on using my money for
paying those really fast and friendly authorities."}

Obviously, this hardens the task of sentiment analysis. Moreover,it is hard to
teach a machine patterns like sarcasm and to identify the intended meaning
behind words.  \cite{liu2010sentimentanalysis,pang2008opinion}


Another crucial point are dependencies of sentiments: topic, domain, and
temporal dependency. Those mainly focus on the problem, that sentiments can
have a different meaning, depending on the underlying topic or domain. E.g. a
word such as "unpredictable" may have a positive meaning if it used for movie
review, but could have a negative sentiment if it used to describe the behavior
of a car. Temporal dependency deals with training a classifier with data from a
certain time-period and use this classifier for data of another time-period.
This may may have an influence on the accuracy of classification as well.
\cite{read2005using}

\subsection{Twitter-based Data} Twitter is a form of microblog, where users can
post small text immediately and the so called tweets are contain real-time
reactions to certain events. The social network platform is categorized as
microblog as every tweet is limited to 140 signs. This results in people using
e.g. abbreviations, emoticons, (intentional) spelling mistakes in order to fit
and express their opinion accurate. Moreover, Twitter uses some special
characters like the \emph{@} which indicates that the post directed to another
user. In addition hashtags are used to refer to special topic.  Another problem
is that Twitter data may contain spam. All these special characteristics play a
huge role when analyzing tweets. \cite{agarwal2011sentiment, read2005using}


\section{Twitter API} \label{twitterapi} Tweets has become to a popular
resource for sentiment analysis, as Twitter provides an API to retrieve tweets
and therefore makes the collection of data easy.


\section{Preprocessing Tweets} \label{preprocessing} In order to achieve the
most precise result when classifying tweets, some preprocessing steps are
suggested.




\section{Classifying Tweets} \label{classification}

\subsection{Corpus} The corpus is the starting point of each sentiment
analyses. It contains the data the will be used to train a classifier.


\subsection{Training}

\subsection{Evaluation}

\section{Related Work} \label{related}


\section{Conclusion} \label{conclusion}

%\end{document}  % This is where a 'short' article might terminate



%
% The following two commands are all you need in the
% initial runs of your .tex file to
% produce the bibliography for the citations in your paper.
\bibliographystyle{abbrv}
\bibliography{sigproc}  % sigproc.bib is the name of the Bibliography in this case
% You must have a proper ".bib" file
%  and remember to run:
% latex bibtex latex latex
% to resolve all references
%
% ACM needs 'a single self-contained file'!
%
%APPENDICES are optional
%\balancecolumns

\end{document}
