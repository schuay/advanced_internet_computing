\documentclass{acm_proc_article-sp}
\usepackage{url}
\usepackage{amsmath}

\title{Twitter-based Sentiment Analysis}

\numberofauthors{5}
\author{
\alignauthor Jakob Gruber\\
      0203440
\alignauthor Matthias Krug\\
      0828965
\alignauthor Stefanie Plieschnegger\\
      0926102
\and
\alignauthor Christian Proske\\
      1328245
\alignauthor Mino Sharkhawy\\
      1025887
}

\begin{document}

\maketitle

\begin{abstract} Sentiment analysis has become hugely popular in recent years.
    Especially Twitter provides a wealth of data for a variety of topics, which
    can be processed and classified in order to provide an approximate measure of
    the overall opinion. However,
    classification of Twitter-based data is in some ways different to traditional
    text mining and introduce additional challenges. In this paper, the
    typical steps and problems of classifying tweets are outlined including
    preprocessing steps, training, and evaluation.
\end{abstract}

\category{H.3}{Information Systems}{Information Search and Retrieval}

\keywords{Sentiment Analysis, Opinion Mining, Twitter, Classifier, Natural
Language Processing}

\section{Introduction}

The general opinion about a specific product or service
has a great influence on its reputation. People often want to know
what others think about a special product they might be willing to buy, about a new
movie, or about a hotel they are going to book. Companies may also be
interested in its customers' opinions, politicians may wish to receive
feedback, and social organizations may have interest in an ongoing debate \cite{pak2010twitter}.
The world wide web provides many ways for people to
distribute their experiences and sentiments. Machine learning algorithms make
it easier to process and evaluate those sentiments and are therefore able to
provide an overall opinion to a certain topic. This type of analysis is called
opinion mining or sentiment analysis \cite{liu2010sentimentanalysis,
pang2008opinion}. Clearly, there are some challenges when assessing the opinion
of people, especially when classifying microblogging services like Twitter\footnote{
\url{http://twitter.com}}. The underlying paper gives an overview
of different sentiment analysis approaches and outlines special problems
related to the classification of microblogging services. These challenges are
discussed in section~\ref{analyzingdata}. In section~\ref{preprocessing} we describe
transformations of tweets and feature selection.
Section~\ref{classification} deals with training and evaluation of classifiers. In
section~\ref{related} we introduce related work, and
section~\ref{conclusion} concludes the paper.


\section{Analyzing Data} \label{analyzingdata}

The analysis of text
and the classification whether its content should be considered positive, negative
or neutral, is the core functionality in sentiment analysis. We
first discuss general considerations of classifying data and subsequently
outline additional challenges related to Twitter-based data.

\subsection{Text Mining vs. Sentiment Analysis}

One would assume that text
mining and sentiment analysis are very similar. Text mining e.g. may deal with
classifying documents by topic, which represents one of the easier tasks.
Topic-based text classification tries to match a text into a category like
sport, politics etc. and therefore topic-related words are identified in order to
classify a text. However, sentiment analysis requires focus on typical
sentiment words such as ``hate'', ``love'', ``like'', ``regret''.  When it comes to
identifying the overall sentiment, it turns out that a text may contain
both positive and negative sentiment words:

\textit{"My new smartphone is really cool, the display is just gorgeous. The
battery life is really bad, however."}

It may even contain positive words within
a generally negative meaning (sarcasm):

\textsl{"Oh, of course - I have a lot of time. Just keep on using my money for
paying those really fast and friendly authorities."}

Obviously, this makes the task of sentiment analysis more difficult.
Moreover, it is hard to
teach a machine patterns such as sarcasm or how to identify the intended meaning
behind words \cite{liu2010sentimentanalysis,pang2008opinion}.


Another crucial point are dependencies; \emph{Topic and domain dependencies}
mainly focus on the problem that sentiments can
have a different meaning depending on the underlying topic or domain.
E.g. a
word such as "unpredictable" may have a positive meaning if it is used in a movie
review, but could have a negative sentiment when used to describe the behavior
of a car. \emph{Temporal dependencies} describe the problem of training a classifier
with data from a certain time-period and applying it for data classification
of another time-period.  These dependencies may have an influence on the
accuracy of classification as well \cite{read2005using,pang2008opinion}.


\subsection{Twitter-based Data} \label{twitter-based}

Twitter is a form of
microblog: users can post short messages of up to 140 characters, often in reaction
to certain events. This results
in people using abbreviations, emoticons, slang and (intentional) spelling
mistakes in order to fit and express their opinion accurately. For example, users
sometimes emphasize certain words by using uppercase characters or
repeating vowels (e.g. "I'm feeling happyyyyy"). Moreover, Twitter uses 
special characters such as the ``@'' (targets direct the post to a specific user)
and ``\#'' (hashtags signify that the post refers to a certain topic).
Another problem is that Twitter data may contain spam. The
fact that tweets can be retweeted should also be kept in mind and may have an
influence on the classifier, depending on the strategy of classification.  All
these special characteristics play a huge role when analyzing tweets
\cite{agarwal2011sentiment, read2005using}.


\section{Preprocessing Tweets} \label{preprocessing}

In order to achieve the
most precise result when classifying tweets, some preprocessing steps are
suggested. Preprocessing describes the cleansing of data and the text
preparation for classification. It needs to be remarked that not all steps are
necessary and it may also depend on the data set and the selected classifier
how much influence these preprocessing tasks have.  Generally, preprocessing
contains data-cleansing steps, so called \texttt{transformations} and feature
selection, also called \texttt{filtering} \cite{haddi2013therole}.


\subsection{Transformation}

Transformation contains all steps that make the
data easier to classify: stripping whitespace, normalization (e.g. to
lower case the text), methods for dealing with noisy data etc.  In the
following, some of these approaches from \cite{ting2011naive,
pak2010twitter,go2009twitter,agarwal2011sentiment,pang2008opinion,haddi2013therole}
are briefly described.

\subsubsection{Extract Noisy Data}

The extraction of noisy data includes
the removal of e.g. advertisement/spam.

\subsubsection{Stopword Removal}

One of the most popular preprocessing steps is
stopword filtering. Stopwords are defined as words that do not contain
additional sentimental information and can therefore be removed from the text,
as it will only make the text shorter, but does not lead to information loss.
Such stopwords are for example: ``a'', ``the'', ``about'', and ``is''.

\subsubsection{Stemming}

This approach deals with the identification of words
that have a similar or identical meaning but are not spelled the same due to
grammatically reasons, e.g. identifying ``was'' as a form of ``be''.

\subsubsection{Emoticon Dictionary}

Emoticons in tweets may be replaced with
its actual meaning. This approach requires a list with all emoticons and its
interpretation. Then those could be labeled according classifications like
extremely-positive, positive, neutral, negative, extremely-negative as
suggested in \cite{agarwal2011sentiment}.

\subsubsection{Stripping Emoticons}

However, the approach of stripping out
emoticons of the training data has also been suggested: \cite{go2009twitter}
consider emoticons as noisy data and experienced a better performance for
training maximum entropy modeling (MaxEnt) and support vector machine (SVM)
classifiers, although the test data may include emoticons.

\subsubsection{Acronym Dictionary}

This preprocessing approach deals with the
use of likely abbreviations in Twitter-data.  Typical acronyms in tweets are
e.g. ``lol'' (laugh out loud), ``brb'' (be right back), ``gr8'' (great) etc.

\subsubsection{Replacing URLs and Targets}

Another preprocessing approach is
replacing all URLs in tweets with a special tag --- this way the actual URL will
not have an influence when classifying the data, only the fact that there is a
URL will have an impact.  The same can be done with targets (already mentioned
in section~\ref{twitter-based}).

\subsubsection{Replace Negations}

The replacement of all negative words like
``non'' or ``never'' by a tag ``not'' ease the classification as well.

\subsubsection{Replacing Repeated Characters}

As pointed out in
section~\ref{twitter-based}, repeated letters are sometimes used to emphasize words.
In order to make these words comparable they may be normalized
by replacing all characters that are repeated more than two
(suggested in \cite{go2009twitter}) or more than three times (suggested in
\cite{agarwal2011sentiment}). The word  ``happyyyyy'' would become ``happyy''
(respectively ``happyyy'' if using the three-times-replacing approach). The
strategy of replacing sequences by three characters makes the use of emphasized
and normal words distinguishable.

\subsection{Feature Selection}

Features are those words or phrases of a text
that will be selected to train a classifier and are  supposed to carry more
sentiment or provide additional information. The frequency of words may have
an influence, or opinion words. % TODO: Huh??
Some approaches are described in the following.

\subsubsection{Tokenization and N-grams}

The data needs to be separated in
order to use the words as features. Normally, the text is split by spaces and
punctuation marks. In addition there are approaches to keep words like "don't"
as one word \cite{pak2010twitter}. Tokenized words are also known as unigrams.
Using n-grams means that combinations of words are used. Unigrams are therefore
combined, depending on the \texttt{n}. Approaches include combining
unigrams and bigrams as features \cite{liu2010sentimentanalysis,
go2009twitter}.

\subsubsection{Part of Speech}

Part of speech (POS) tags deal with the
syntactic analysis of sentences. For example, adjectives are assumed to provide
sentimental meaning \cite{liu2010sentimentanalysis}.
POS taggers require adaption to the specific properties of tweets.
One such approach can be found in \cite{gimpel2011part}. 

\subsubsection{Opinion Words}

Some words such as ``love'', ``hate'', ``beautiful'', ``great'' are 
known to carry special sentiment.
Also, phrases like \texttt{"All that glitters
isn't gold."} may be associated with special meaning during feature selection
\cite{liu2010sentimentanalysis}.

\subsubsection{Twitter Specifics}

% TODO: What?? This directly contradicts the paragraph from before describing
% how URLs are stripped from tweets.
Obviously URLs, emoticons, hashtags and
targets are carrying special meaning in each tweet and are therefore also
considered as appropriate features \cite{gimpel2011part}.

\section{Classifying Tweets} \label{classification}

Twitter has become a
popular resource for sentiment analysis, as it provides a REST
API\footnote{\url{https://dev.twitter.com/docs/api/1.1}} and a stream
API\footnote{\url{https://dev.twitter.com/docs/streaming-apis}} to retrieve tweets
and therefore makes the collection of data easy.  In this section the basic
approach of training and testing classifiers will be outlined.


\subsection{Data Set}

The corpus is the starting point of each sentiment
analysis. It contains the data that will be used to train a classifier and
therefore should contain appropriate information. % TODO: Which information??
For supervised machine
learning approaches the data set needs to be labeled, i.e. each training element
must be assigned a sentiment category (such as positive, negative, or neutral).
Many data sets are available as outlined in \cite{kouloumpis2011twitter}. It is also
possible to collect own data as suggested in \cite{pak2010twitter}.


\subsection{Training}

Both unsupervised and
supervised classifiers are in use. Unsupervised ones rely on signal words, typical
phrases, and patterns of POS tags. See
\cite{liu2010sentimentanalysis} and \cite{pang2008opinion} for more details.

However, most sentiment classifications are based on supervised learning which
requires a labeled data set (e.g. positive and negative), which will be
separated into a training set and a testing set. Among the most popular
classifiers in sentiment analysis are the Naive Bayes and Support Vector Machine (SVM) algorithms. 

Naive Bayes is a simple algorithm that generally performs well in sentiment
analysis domains. It calculates the likeliness that one object belongs to a
class. It has been shown that preprocessing and feature selection play an
important role in order to improve the accuracy of naive bayes \cite{ye2009sentiment, ting2011naive}.
% TODO: Ok, but isn't that the case with every classifier?

SVM classifiers usually perform better than Naive Bayes.
% TODO: Some numbers? Data?
Their approach is to separate the
positive and negative training vectors of the data set with a maximum margin
\cite{ye2009sentiment}.

\subsection{Experiments and Evaluation}

The evaluation of classifiers is the
same as for traditional machine learning algorithms. In order to find the best
classifier, the preprocessing steps described in section~\ref{preprocessing}
are usually used in different combinations. Moreover, $k$-fold-cross-validations
is a common approach where data is split into $k$ folds, using $k-1$ folds as
training data and $1$ as testing set, repeated $k$ times so that each fold will be
used as testing set once. 

Typically the most important benchmark figures are accuracy, precision, recall,
and F-measure. The number of recognized true positives (TP), true negatives
(TN), as well as false negatives (FN) and  false positives (FP) are used as follows \cite{haddi2013therole, ting2011naive,
sokolova2006beyond, pak2010twitter, ye2009sentiment}:

\begin{align*}
\text{Accuracy} &= \frac{TP+TN}{TP+TN+FN+FP} \\
\text{Precision} &= \frac{TP}{TP+FP} \\
\text{Recall} &= \frac{TP}{TP+FN} \\
\text{F-measure} &= \frac{2 \cdot \text{Recall} \cdot \text{Precision}}{\text{Recall}+\text{Precision}}
\end{align*}


\section{Related Work} \label{related}

Sentiment analysis of Twitter data
has been the focus of various researchers. One of the results, described in
\cite{agarwal2011sentiment} claims to have an average accuracy of around
60-75~\% (depending on the selected features and labels).  The approach of
testing different machine learning algorithms (naive bayes, MaxEnt and SVM)
combined with different features in \cite{go2009twitter} revealed an average
accuracy of 80~\%, however. This is underlined by the case study in
\cite{lin2012large}, which experienced a similar result.

\section{Conclusion} \label{conclusion}

When analyzing tweets, specific
characteristics like the limited size, slang, hashtags, targets, etc. must be
considered. We have listed typical preprocessing steps for Twitter data and
provided an overview of classification approaches.  Since Twitter provides free
access to its data, and people or organizations are interested in aggregated
opinions, it is likely that sentiment analysis of tweets will become an even
more popular research area in the future.

\bibliographystyle{abbrv}
\bibliography{sigproc}

\end{document}
